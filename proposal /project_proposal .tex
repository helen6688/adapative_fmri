\documentclass[12pt]{article}
\usepackage{fullpage,graphicx,psfrag,amsmath,amsfonts,verbatim}
\usepackage[small,bf]{caption}
\usepackage{hyperref}


\bibliographystyle{alpha}

\title{Discovering signals in fMRI data; 
a multiple-testing, Bayesian nonparametric approach   \\ \large{Project Proposal for STAT308}}
\author{Ahmed Bou-Rabee, Wanrong Zhu, Zheng Xu, Mo Zhou}

\begin{document}
\maketitle

{\bf Introduction } 

The goal of this project is to formulate and test a method which can be used to adaptively identify 
clusters of signals in functional magnetic resonance imaging (fMRI) data. 
Roughly, fMRI measures the change in brain blood flow associated 
with mental activity \cite{huettel2004functional}. The brain is divided into regions known as voxels, and the intensity 
of the blood flow over each voxel is recorded at evenly spaced time intervals while the subject is stimulated. The data is then in the form 
(voxel, time, intensity of reading). For example, suppose researchers wanted to identify regions 
of the brain associated with hunger or craving. To aid in this, fMRI readings can be taken while subjects are shown pictures of pizza and hamburgers. 

An advantage of using fMRI is that it's a noninvasive procedure.  Due to this, there are many publicly available datasets \cite{poldrack2013toward}.
However, analyzing fMRI data poses many statistical challenges, one of them being the multiple comparisons problem. 
 Any sampling procedure involves error.  Because there are typically thousands of voxels, it's likely that regions 
 may have high readings, but not be statistically significant.  Also, nearby readings are likely correlated.  If researchers 
 wanted to identify clusters of voxels which are significant over time, this introduces an additional challenge. 
 
 
 XXX  explain what previous work has done.  e.g. reference Rina's paper \cite{foygel2015p} 

 For our purposes, we are going to assume that the intensity of the reading is given to us in the form of a p-value between 0 and 1. 
 This corresponds to the hypothesis that 
 at time i voxel j is significant . Just like Rina's paper. 
 
 This has been done in the literature
 
   

In our project, we will focus on two questions: 
\begin{itemize}
\item Can we adaptively identify entire regions of the brain (not just voxels) which are associated with the stimulus while
accounting for multiple testing error?
\item There is no response variable in fMRI data, so how can we reliably test our algorithm? 
\end{itemize}

Here is an outline on how we'll tackle these two topics. 

{\bf Adaptively identifying regions of interest} 
Research has already worked on how to identify significant voxels of interest and also regions if those
regions are already given to the statistician.  One particular issue we want to focus on 
is identifying "regions of interest" adaptively, instead of using predefined ones.  
Here's one idea we have for doing so. 

If we assume that the data is generated according to a Hierarchial Bayesian 
process and then identify the posterior distribution, we automatically answer the question
without any multiple testing problems. 

We assume that the data is generated in the following way. 
\begin{enumerate}
\item  For each location $\zeta_i$, generate a prior distribution $F_i$ which is drawn 
from a prior $\Phi$ on the set of prior distributions which have clusters. 


\item For each location $\zeta_i$ generate the intensity according to the prior $F_i$ 
\end{enumerate}



Once we have identified prior distributions, we identify clusters of high activity 
by picking clusters which have a distribution that is sufficiently far
away from uniform.  Kullback?Leibler divergence from uniform 


{\bf Testing the algorithm } 

Second question: 
Resample from the data and see how "different" nonparametric algorithm is. 




What datasets will we be working with?  Do we have any baselines for comparison? 
First test will be Rina's data. 


 
\bibliography{bibliography}



\end{document}


